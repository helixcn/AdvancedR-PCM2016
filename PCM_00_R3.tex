\documentclass[UTF8]{beamer} 
\usepackage{ctex}
\usepackage{Sweave}
\usetheme{Boadilla} 
\usepackage{color} 
\usepackage{graphicx} 

\begin{document}

\title{R编程与进化分析} 
\subtitle{内容与目标}
\author{张金龙} 
\institute{jinlongzhang01@gmail.com}
\date{2016年5月9日 北京} 


%%% Title Page 
\frame{\titlepage} 

%%% Page one
%\begin{frame}[fragile]
%\frametitle{开始检验}
%\begin{Sinput}
%># 中文注释, 本句并不运行
%>x1 <- runif(100)
%\end{Sinput}
%\end{frame}

\section*{Outline}
\begin{frame}
\frametitle{目录}
\tableofcontents
\end{frame}

\section{进化树之后要回答的问题}
%\begin{frame}
%\frametitle{目录}
%\tableofcontents[currentsection] 
%\end{frame}


\begin{frame}
\frametitle{进化树能帮我们回答哪些问题?}
\begin{itemize}
\item 1. 分类单元之间的系统发育关系
\item 2. 物种的形成速率受什么影响? 如: 类群的古老程度, 类群的丰富度, 类群所处的纬度, 类群的特殊生境
\item 3. 物种形成速率和灭绝速率在历史上发生过哪些变化?
\item 4. 物种的进化历史越独特, 越应该受到保护吗?
\item 5. 相近的物种有相似的性状吗? 有相似的习性吗?
\end{itemize}
\end{frame}


\begin{frame}
\frametitle{进化树能帮我们回答哪些问题?}
\begin{itemize}
\item 6. 如果已知某一分支的性状, 是否能够了解其祖先的性状? 
\item 7. 物种的适应性是如何进化的? 
\item 8. 已知物种的当前分布区, 如何获得其祖先分布区?
\item 9. 群落内物种的组成是随机的, 还是由于对生境的偏好造成的?
\end{itemize}
\end{frame}

\section{本领域主要学者}
\begin{frame}
\frametitle{系统发育比较分析的主要学者}
\begin{itemize}
\item M. Donoghue,L. Harmon,
\item A. Purvis, A. Rambaut 
\item D. Sluter, J. Weir
\item R. E. Ricklefs
\item J. J. Wiens
\item R. A. Pyron
\item J. Losos, W. Jetz   J.L. Gittleman
\end{itemize}
\end{frame}

\begin{frame}
\frametitle{系统发育比较分析的主要学者}
\begin{itemize}
\item T. Garland, S. Blomberg
\item D. Ackerly, Cam Webb
\item J. Felsenstein
\item D. Maddison \& W. Maddison \& 
\item N. Swenson, N. Kraft, Marc Cadotte
\item R. Freckleton, M. Pagel, E. Paradis 
\item D. Radobsky, R. FitzJohn
\item R. Ree, S. Smith
\end{itemize}
\end{frame}

\section{掌握CRAN Task View上的工具}
\begin{frame}
\frametitle{R CRAN Task Views 的内容 I }

\begin{itemize}
\item 基于序列或者植物名录建立进化树 Phylogenetic Inference 
\item 进化树导入R Getting trees into R
\item 进化树基本调整 Utility functions: eg. resolving/ladderize
\item 进化树的基本操作 Tree Manipulation
\item 进化树绘制 Tree Plotting and Visualization
\item 祖先状态重建 Ancestral state reconstruction
\item 物种分化分析 Diversification Analysis
\item 分子钟 Divergence Times
\item 进化树的模拟 Tree Simulations
\end{itemize}  
\end{frame}

\begin{frame}
\frametitle{R CRAN Task Views 的内容 II}

\begin{itemize}
\item 性状进化 Trait evolution
\item 性状模拟 Trait Simulations
\item 群落系统发育 Community/Microbial Ecology
\item 气候适应性进化 Phyloclimatic Modeling
\item 重建祖先分布区 Phylogeography/Biogeography
\item 物种与种群的界定与模拟 Species/Population Delimitation

\item Taxonomy
\end{itemize}  
\end{frame}

\section{学习目标}
\begin{frame}
\frametitle{本课程能学到什么?}
\begin{itemize}
\item 搭建开发平台:R程序包的维护,CRAN, R-forge, Github, Bioconductor, 版本控制 git
\item R语言的基本用法,对象,编写函数
\item 文本处理与大数据,正则表达式
\item S3 与 S4,Rcpp与其他语言混合编程
\item R编程的习惯
\item 建立进化树,与分子钟校对
\item 进化树读取以及基本操作: ape
\end{itemize}
\end{frame}


\begin{frame}
\frametitle{本课程能学到什么?}
\begin{enumerate}
\item R绘图基础,以及绘制进化树 ape, ggplot2, ggtree
\item 系统发育比较分析的基本原理 
\begin{itemize}
\item 极大似然分析
\item 贝叶斯推断
\item 优化方法
\item 随机化零模型
\item 模型选择
\item Bootstrap
\end{itemize}
\end{enumerate}
\end{frame}

\begin{frame}
\frametitle{本课程能学到什么?}
\begin{itemize}
\item 系统发育多样性与进化独特性
\item 群落系统发育 Phylomatic, Phylocom , picante
\item 分化速率与灭绝速率推断, diversitree, laser, geiger, MEDUSA
\end{itemize}
\end{frame}

\end{document}
